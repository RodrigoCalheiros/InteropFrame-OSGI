\chapter{Considera��es Finais}\label{chap:consideracoes}

O InteropFrame � uma solu��o que auxilia na resolu��o de problemas relacionados � interoperabilidade entre componentes distribu�dos de modelos heterog�neos. Por�m, essa solu��o est� apenas limitada aos modelos OpenCOM e Fractal, al�m de promover a comunica��o remota apenas utilizando Java RMI ou \textit{Web Service SOAP}. Outra limita��o � que o InteropFrame foi desenvolvido em Java ``puro'', o que dificulta a sua extensibilidade. Outro problema do InteropFrame � a sua comunica��o interna (entre os lados cliente e servidor), que � feita atrav�s de Java RMI.

Este trabalho prop�e uma poss�vel solu��o para essa limita��o atrav�s da utiliza��o do OSGi como forma de modularizar o InteropFrame, al�m da extens�o para o OSGi como um modelo de componentes interoper�vel dentro da ferramenta. Tamb�m prop�e a resolu��o dos problemas relacionados � comunica��o interna atrav�s dos mecanismos de comunica��o providos pelo \textit{Eclipse Communication Framework}. Dessa forma, o InteropFrame torna-se mais extens�vel, coeso e desacoplado.