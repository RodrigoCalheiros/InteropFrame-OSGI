%Elementos Pré-textuais

%% ---------------------------------------------------------------------------- %
%% CAPA

\thispagestyle{empty}
\begin{titlepage}
    \begin{figure*}
        \centering
        \includegraphics[keepaspectratio = true, scale=1]{ufs.jpg} \label{fig:logoUFS}
    \end{figure*}
    \vskip 1.5 cm
    \begin{center}
        {\Large UNIVERSIDADE FEDERAL DE SERGIPE \\ CENTRO DE CI\^ENCIAS EXATAS E TECNOL\'OGICAS \\ DEPARTAMENTO DE COMPUTAC\~AO }\\
        \vspace{6cm}
        {\bf \LARGE Modularização e Extensão do InteropFrame}\\
        \vspace{3cm}
        {\large Felipe Oliveira Carvalho}\\
				{\large Rodrigo Losano Fontes Calheiros}\\
        \vfill
        \large{S\~ao Crist\'ov\~ao -- SE\\ 2014}
    \end{center}
\end{titlepage}

%% ---------------------------------------------------------------------------- %
%% Contracapa
%\newpage
%\thispagestyle{empty}
%\onehalfspacing
%\begin{titlepage}
%    \begin{center}
%        {\Large Adonis Vieira de Andrade}\\
%        \vspace{10cm}
%        {\bf \LARGE Aloca\c c\~ao \'O}\\
%        \vfill
%        \large{S\~ao Crist\'ov\~ao -- SE\\ 2013}
%    \end{center}
%\end{titlepage}
%\pagebreak

%% ---------------------------------------------------------------------------- %

% Folha de rosto
\newpage
\thispagestyle{empty}
\begin{titlepage}
    \begin{center}
        {\Large Felipe Oliveira Carvalho \\ Rodrigo Losano Fontes Calheiros }\\[8.3cm]
        {\LARGE Modularização e Extensão do InteropFrame} \\[4.3cm]
        \hspace{.45\textwidth} % posicionando a minipage
        \begin{minipage}{.5\textwidth}
            \singlespacing
                Pré- Projeto do Trabalho de Conclus\~ao de Curso apresentado como requisito parcial para obten\c c\~ao do t\'itulo de Bacharel em Sistemas de Informação da Universidade Federal de Sergipe. \\
                Orientador: Prof. Dr. Tarc\'isio da Rocha
            \onehalfspacing
        \end{minipage}
        \vfill
        {\large S\~ao Crist\'ov\~ao -- SE} \\
        {\large 2014}
    \end{center}
\end{titlepage}
\pagebreak

%-----Folha de aprovação
\newpage
\thispagestyle{empty}
\begin{titlepage}
    \begin{center}
        {\LARGE Modularização e Extensão do InteropFrame}\\[2.3cm]
        {\Large Felipe Oliveira Carvalho \\ Rodrigo Losano Fontes Calheiros} \\[1.5cm]
        \hspace{.45\textwidth} % posicionando a minipage
        \begin{minipage}{.5\textwidth}
            \singlespacing
                Pré- Projeto do Trabalho de Conclus\~ao de Curso apresentado como requisito parcial para obten\c c\~ao do t\'itulo de Bacharel em Sistemas de Informação da Universidade Federal de Sergipe. \\

            \onehalfspacing
        \end{minipage}\\[2.5cm]

        \centerline{Aprovado em: \underline{ }\underline{ }\underline{ }\underline{ }\underline{ }\underline{ } /\underline{ }\underline{ }\underline{ }\underline{ }\underline{ }\underline{ } /\underline{ }\underline{ }\underline{ }\underline{ }\underline{ }\underline{ }.}
        \centerline{}
        \centerline{}
        \centerline{Banca Examinadora:}

        \centerline{}
        \centerline{\underline{ }\underline{ }\underline{ }\underline{ }\underline{ }\underline{ }\underline{ }\underline{ }\underline{ }\underline{ }\underline{ }\underline{ }\underline{ }\underline{ }\underline{ }\underline{ }\underline{ }\underline{ }\underline{ }\underline{ }\underline{ }\underline{ }\underline{ }\underline{ }\underline{ }\underline{ }\underline{ }\underline{ }\underline{ }\underline{ }\underline{ }\underline{ }\underline{ }\underline{ }\underline{ }\underline{ }\underline{ }\underline{ }\underline{ }\underline{ }\underline{ }\underline{ }\underline{ }\underline{ }\underline{ }\underline{ }\underline{ }\underline{ }\underline{ }\underline{ }\underline{ }\underline{ }\underline{ }\underline{ }\underline{ }\underline{ }\underline{ }\underline{ }\underline{ }\underline{ }\underline{ }\underline{ }\underline{ }\underline{ }\underline{ }\underline{ }\underline{ }\underline{ }\underline{ }\underline{ }\underline{ }\underline{ }\underline{ }\underline{ }\underline{ }\underline{ }\underline{ }}
        \centerline{Prof. Dr. Tarc\'isio da Rocha (Orientador)}
        \centerline{Universidade Federal de Sergipe}
        \centerline{}

        \centerline{\underline{ }\underline{ }\underline{ }\underline{ }\underline{ }\underline{ }\underline{ }\underline{ }\underline{ }\underline{ }\underline{ }\underline{ }\underline{ }\underline{ }\underline{ }\underline{ }\underline{ }\underline{ }\underline{ }\underline{ }\underline{ }\underline{ }\underline{ }\underline{ }\underline{ }\underline{ }\underline{ }\underline{ }\underline{ }\underline{ }\underline{ }\underline{ }\underline{ }\underline{ }\underline{ }\underline{ }\underline{ }\underline{ }\underline{ }\underline{ }\underline{ }\underline{ }\underline{ }\underline{ }\underline{ }\underline{ }\underline{ }\underline{ }\underline{ }\underline{ }\underline{ }\underline{ }\underline{ }\underline{ }\underline{ }\underline{ }\underline{ }\underline{ }\underline{ }\underline{ }\underline{ }\underline{ }\underline{ }\underline{ }\underline{ }\underline{ }\underline{ }\underline{ }\underline{ }\underline{ }\underline{ }\underline{ }\underline{ }\underline{ }\underline{ }\underline{ }\underline{ }}
        \centerline{Prof. xxxxx xxxxxxx xxxxxxx}
        \centerline{Universidade Federal de Sergipe}
        \centerline{}

        \centerline{\underline{ }\underline{ }\underline{ }\underline{ }\underline{ }\underline{ }\underline{ }\underline{ }\underline{ }\underline{ }\underline{ }\underline{ }\underline{ }\underline{ }\underline{ }\underline{ }\underline{ }\underline{ }\underline{ }\underline{ }\underline{ }\underline{ }\underline{ }\underline{ }\underline{ }\underline{ }\underline{ }\underline{ }\underline{ }\underline{ }\underline{ }\underline{ }\underline{ }\underline{ }\underline{ }\underline{ }\underline{ }\underline{ }\underline{ }\underline{ }\underline{ }\underline{ }\underline{ }\underline{ }\underline{ }\underline{ }\underline{ }\underline{ }\underline{ }\underline{ }\underline{ }\underline{ }\underline{ }\underline{ }\underline{ }\underline{ }\underline{ }\underline{ }\underline{ }\underline{ }\underline{ }\underline{ }\underline{ }\underline{ }\underline{ }\underline{ }\underline{ }\underline{ }\underline{ }\underline{ }\underline{ }\underline{ }\underline{ }\underline{ }\underline{ }\underline{ }\underline{ }}
        \centerline{Prof. xxxxxx xxxxxxx xxxxxx}
        \centerline{Universidade Federal de Sergipe}
        \centerline{}

        \vfill
        {\large S\~ao Crist\'ov\~ao -- SE} \\
        {\large 2014}
    \end{center}
\end{titlepage}
\pagebreak

%-----FIm da folha de aprovação


\pagenumbering{roman}     % começamos a numerar

% ---------------------------------------------------------------------------- %
% Agradecimentos:
% Se o candidato não quer fazer agradecimentos, deve simplesmente eliminar esta página

%\chapter*{Agradecimentos}
%
%Agrade\c co aos meus pais pela oportunidade que me deram de estudar e pelo carinho e apoio em todos os momentos.
%
%\`A minha namorada, Monique, por todo o incentivo e apoio nas horas dif\'iceis.
%
%Aos meus amigos, em especial a turma 2007, pela amizade e companheirismo durante todos esses anos.
%
%Ao meu orientador Professor \^Angelo M. F. de Almeida pelo tempo disponibilizado para me ajudar e guiar ao longo deste trabalho.

% ---------------------------------------------------------------------------- %

% Resumo
\chapter*{Resumo}




\noindent \textbf{Palavras-chave:} Modelos de Componentes, Modularização, Sistemas Distribuídos.

% ---------------------------------------------------------------------------- %
% Abstract
%\chapter*{Abstract}
%
%This paper deals with the optimal allocation of telecontrolled switches and its main advantages in a distribution system feeder. This problem is solved using a Genetic Algorithm which has as its fitness function minimize the losses by energy not supplied. A real feeder is modeled and its continuity index are calculated using a logical structural matrix. The load flow calculation is made by applying the Power Summation Method. The implemented Genetic Algorithm is validated through the allocation of 2 telecontrolled switches in a real distribution system feeder. Advantages of allocation of 2 and 3 telecontrolled switches are compared and analyzed between themselves.
%\\
%
%\noindent \textbf{Keywords:} Genetic Algorithm,optimal allocation of telecontrolled switches, Distribution System  .

% ---------------------------------------------------------------------------- %
% Sumário
\tableofcontents    % imprime o sumário

%%% ---------------------------------------------------------------------------- %
\chapter{Lista de Abreviaturas}
\begin{tabular}{ll}
 %   AG  	&	Algoritmos Gen\'eticos	\\
 %   ANEEL	&	Ag\^encia Nacional de Energia El\'etrica	\\
@OP & Attribute Oriented Programming \\
ADL & Architecture Description Language \\
API & Application Programming Interface \\
IDE & Integrated Development Enviroment \\
J2EE & Java 2 Enterprise Edition \\
OSGi & Open Services Gateway Initiative \\
SAVECCM & SAVE Comp Component Model \\
UML & Unified Markup Language \\
XML & Extensible Markup Language \\



		 
   
\end{tabular}

% ---------------------------------------------------------------------------- %


% Listas de figuras e tabelas criadas automaticamente
\listoffigures
%\listoftables


