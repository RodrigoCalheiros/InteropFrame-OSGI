%Elementos Pré-textuais

%% ---------------------------------------------------------------------------- %
%% CAPA

\thispagestyle{empty}
\begin{titlepage}
    \begin{figure*}
        \centering
        \includegraphics[keepaspectratio = true, scale=1]{ufs.jpg} \label{fig:logoUFS}
    \end{figure*}
    \vskip 1.5 cm
    \begin{center}
        {\Large UNIVERSIDADE FEDERAL DE SERGIPE \\ CENTRO DE CI\^ENCIAS EXATAS E TECNOL\'OGICAS \\ DEPARTAMENTO DE COMPUTA\c{C}\~AO }\\
        \vspace{6cm}
        {\bf \LARGE Modulariza\c{c}\~ao e Extens\~ao do InteropFrame}\\
        \vspace{3cm}
        {\large Felipe Oliveira Carvalho}\\
				{\large Rodrigo Losano Fontes Calheiros}\\
        \vfill
        \large{S\~ao Crist\'ov\~ao -- SE\\ 2014}
    \end{center}
%\end{titlepage}

%% ---------------------------------------------------------------------------- %
%% Contracapa
%\newpage
%\thispagestyle{empty}
%\onehalfspacing
%\begin{titlepage}
%    \begin{center}
%        {\Large Adonis Vieira de Andrade}\\
%        \vspace{10cm}
%        {\bf \LARGE Aloca\c c\~ao \'O}\\
%        \vfill
%        \large{S\~ao Crist\'ov\~ao -- SE\\ 2013}
%    \end{center}
%\end{titlepage}
%\pagebreak

%% ---------------------------------------------------------------------------- %

% Folha de rosto
\newpage
\thispagestyle{empty}
%\begin{titlepage}
    \begin{center}
        {\Large Felipe Oliveira Carvalho \\ Rodrigo Losano Fontes Calheiros }\\[8.3cm]
        {\LARGE Modulariza\c{c}\~ao e Extens\~ao do InteropFrame} \\[4.3cm]
        \hspace{.45\textwidth} % posicionando a minipage
        \begin{minipage}{.5\textwidth}
            \singlespacing
                Pr\'e- Projeto do Trabalho de Conclus\~ao de Curso apresentado como requisito parcial para obten\c c\~ao do t\'itulo de Bacharel em Sistemas de Informa\c{c}\~ao da Universidade Federal de Sergipe. \\ \\ \\ \\
                
            \onehalfspacing
        \end{minipage}
				{\large Orientador: Prof. Dr. Tarc\'isio da Rocha}
        \vfill
        {\large S\~ao Crist\'ov\~ao -- SE} \\
        {\large 2014}
    \end{center}
%\end{titlepage}
\pagebreak

%-----Folha de aprovação
\newpage
\thispagestyle{empty}
%\begin{titlepage}
    \begin{center}
        {\LARGE Modulariza\c{c}\~ao e Extens\~ao do InteropFrame}\\[2.3cm]
        {\Large Felipe Oliveira Carvalho \\ Rodrigo Losano Fontes Calheiros} \\[1.5cm]
        \hspace{.45\textwidth} % posicionando a minipage
        \begin{minipage}{.5\textwidth}
            \singlespacing
                Pr\'e- Projeto do Trabalho de Conclus\~ao de Curso apresentado como requisito parcial para obten\c c\~ao do t\'itulo de Bacharel em Sistemas de Informa\c{c}\~ao da Universidade Federal de Sergipe. \\

            \onehalfspacing
        \end{minipage}\\[2.5cm]

        \centerline{Aprovado em: \underline{ }\underline{ }\underline{ }\underline{ }\underline{ }\underline{ } /\underline{ }\underline{ }\underline{ }\underline{ }\underline{ }\underline{ } /\underline{ }\underline{ }\underline{ }\underline{ }\underline{ }\underline{ }.}
        \centerline{}
        \centerline{}
        \centerline{Banca Examinadora:}

        \centerline{}
        \centerline{\underline{ }\underline{ }\underline{ }\underline{ }\underline{ }\underline{ }\underline{ }\underline{ }\underline{ }\underline{ }\underline{ }\underline{ }\underline{ }\underline{ }\underline{ }\underline{ }\underline{ }\underline{ }\underline{ }\underline{ }\underline{ }\underline{ }\underline{ }\underline{ }\underline{ }\underline{ }\underline{ }\underline{ }\underline{ }\underline{ }\underline{ }\underline{ }\underline{ }\underline{ }\underline{ }\underline{ }\underline{ }\underline{ }\underline{ }\underline{ }\underline{ }\underline{ }\underline{ }\underline{ }\underline{ }\underline{ }\underline{ }\underline{ }\underline{ }\underline{ }\underline{ }\underline{ }\underline{ }\underline{ }\underline{ }\underline{ }\underline{ }\underline{ }\underline{ }\underline{ }\underline{ }\underline{ }\underline{ }\underline{ }\underline{ }\underline{ }\underline{ }\underline{ }\underline{ }\underline{ }\underline{ }\underline{ }\underline{ }\underline{ }\underline{ }\underline{ }\underline{ }}
        \centerline{Prof. Dr. Tarc\'isio da Rocha (Orientador)}
        \centerline{Universidade Federal de Sergipe}
        \centerline{}

        \centerline{\underline{ }\underline{ }\underline{ }\underline{ }\underline{ }\underline{ }\underline{ }\underline{ }\underline{ }\underline{ }\underline{ }\underline{ }\underline{ }\underline{ }\underline{ }\underline{ }\underline{ }\underline{ }\underline{ }\underline{ }\underline{ }\underline{ }\underline{ }\underline{ }\underline{ }\underline{ }\underline{ }\underline{ }\underline{ }\underline{ }\underline{ }\underline{ }\underline{ }\underline{ }\underline{ }\underline{ }\underline{ }\underline{ }\underline{ }\underline{ }\underline{ }\underline{ }\underline{ }\underline{ }\underline{ }\underline{ }\underline{ }\underline{ }\underline{ }\underline{ }\underline{ }\underline{ }\underline{ }\underline{ }\underline{ }\underline{ }\underline{ }\underline{ }\underline{ }\underline{ }\underline{ }\underline{ }\underline{ }\underline{ }\underline{ }\underline{ }\underline{ }\underline{ }\underline{ }\underline{ }\underline{ }\underline{ }\underline{ }\underline{ }\underline{ }\underline{ }\underline{ }}
        \centerline{Prof. xxxxx xxxxxxx xxxxxxx}
        \centerline{Universidade Federal de Sergipe}
        \centerline{}

        \centerline{\underline{ }\underline{ }\underline{ }\underline{ }\underline{ }\underline{ }\underline{ }\underline{ }\underline{ }\underline{ }\underline{ }\underline{ }\underline{ }\underline{ }\underline{ }\underline{ }\underline{ }\underline{ }\underline{ }\underline{ }\underline{ }\underline{ }\underline{ }\underline{ }\underline{ }\underline{ }\underline{ }\underline{ }\underline{ }\underline{ }\underline{ }\underline{ }\underline{ }\underline{ }\underline{ }\underline{ }\underline{ }\underline{ }\underline{ }\underline{ }\underline{ }\underline{ }\underline{ }\underline{ }\underline{ }\underline{ }\underline{ }\underline{ }\underline{ }\underline{ }\underline{ }\underline{ }\underline{ }\underline{ }\underline{ }\underline{ }\underline{ }\underline{ }\underline{ }\underline{ }\underline{ }\underline{ }\underline{ }\underline{ }\underline{ }\underline{ }\underline{ }\underline{ }\underline{ }\underline{ }\underline{ }\underline{ }\underline{ }\underline{ }\underline{ }\underline{ }\underline{ }}
        \centerline{Prof. xxxxxx xxxxxxx xxxxxx}
        \centerline{Universidade Federal de Sergipe}
        \centerline{}

        \vfill
        {\large S\~ao Crist\'ov\~ao -- SE} \\
        {\large 2014}
    \end{center}
\end{titlepage}
\pagebreak

%-----FIm da folha de aprovação


%\pagenumbering{roman}     % começamos a numerar

% ---------------------------------------------------------------------------- %
% Agradecimentos:
% Se o candidato não quer fazer agradecimentos, deve simplesmente eliminar esta página

%\chapter*{Agradecimentos}
%
%Agrade\c co aos meus pais pela oportunidade que me deram de estudar e pelo carinho e apoio em todos os momentos.
%
%\`A minha namorada, Monique, por todo o incentivo e apoio nas horas dif\'iceis.
%
%Aos meus amigos, em especial a turma 2007, pela amizade e companheirismo durante todos esses anos.
%
%Ao meu orientador Professor \^Angelo M. F. de Almeida pelo tempo disponibilizado para me ajudar e guiar ao longo deste trabalho.

% ---------------------------------------------------------------------------- %

% Resumo
%\chapter*{Resumo}

%O desenvolvimento de Sistemas Distribu\'idos tem se tornado uma tarefa complexa. Uma t\'ecnica que tornou-se amplamente utilizada no desenvolvimento desses sistemas \'e a Engenharia de Software Baseada em Componentes (ESBC), o que resultou no surgimento de diversos modelos de componentes. Um desafio que surge do desenvolvimento baseado em componentes \'e o da interoperabilidade entre partes heterog\^eneas desenvolvidas em diferentes modelos de componentes. Em geral, o problema da interoperabilidade entre partes heterog\^eneas de um sistema \'e tratado pelo uso de \textit{middlewares}. Sistemas de \textit{middleware} s\~ao capazes de promover integra\c{c}\~ao e abstrair do desenvolvedor detalhes dessa integra\c{c}\~ao.

%Dentro desse contexto foi criado o InteropFrame, um \textit{middleware} que lida com a interoperabilidade entre sistemas distribu\'idos desenvolvidos nos modelos de componentes OpenCOM e Fractal, al\'em de tratar de detalhes de comunica\c{c}\~ao remota utilizando RMI e \textit{Web Services SOAP}. O InteropFrame \'e uma solu\c{c}\~ao desenvolvida em Java puro e \'e extens\'ivel para o suporte a novos modelos de componentes e tamb\'em de comunica\c{c}\~ao remota. Por\'em, por ser desenvolvido em Java puro, esse suporte fica dificultado, uma vez que n\~ao s\~ao estabelecidas interfaces modulares para a extensibilidade. Al\'em dessa problem\'atica, o InteropFrame tamb\'em possui limita\c{c}\~oes na comunica\c{c}\~ao remota interna.

%Este trabalho prop\~oe a modulariza\c{c}\~ao do InteropFrame utilizando o modelo OSGi, de forma a reorganizar sua arquitetura em \textit{plug-ins} para os modelos de componentes suportados e tamb\'em para os modelos de comunica\c{c}\~ao remota. Visando confirmar a proposta de modulariza\c{c}\~ao, ser\'a adicionado o suporte ao modelo OSGi para tornar-se interoper\'avel com os modelos j\'a suportados pelo InteropFrame. Al\'em da modulariza\c{c}\~ao, ser\~ao feitas melhorias internas na comunica\c{c}\~ao remota do InteropFrame.
%\newline
%\newline
%\noindent \textbf{Palavras-chave:} Modelos de Componentes, Interoperabilidade, Modulariza\c{c}\~ao, Sistemas Distribu\'idos.

% ---------------------------------------------------------------------------- %
% Abstract
%\chapter*{Abstract}
%
%This paper deals with the optimal allocation of telecontrolled switches and its main advantages in a distribution system feeder. This problem is solved using a Genetic Algorithm which has as its fitness function minimize the losses by energy not supplied. A real feeder is modeled and its continuity index are calculated using a logical structural matrix. The load flow calculation is made by applying the Power Summation Method. The implemented Genetic Algorithm is validated through the allocation of 2 telecontrolled switches in a real distribution system feeder. Advantages of allocation of 2 and 3 telecontrolled switches are compared and analyzed between themselves.
%\\
%
%\noindent \textbf{Keywords:} Genetic Algorithm,optimal allocation of telecontrolled switches, Distribution System  .

% ---------------------------------------------------------------------------- %
% Sumário
%\tableofcontents    % imprime o sumário

%%% ---------------------------------------------------------------------------- %
%\chapter{Lista de Abreviaturas}
%\begin{tabular}{ll}

%IDE & Integrated Development Enviroment \\
%OSGi & Open Services Gateway Initiative \\
%XML & Extensible Markup Language \\		 
   
%\end{tabular}

% ---------------------------------------------------------------------------- %


% Listas de figuras e tabelas criadas automaticamente
%\listoffigures
%\listoftables


